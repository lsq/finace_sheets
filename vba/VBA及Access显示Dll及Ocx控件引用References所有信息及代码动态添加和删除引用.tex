% Options for packages loaded elsewhere
\PassOptionsToPackage{unicode}{hyperref}
\PassOptionsToPackage{hyphens}{url}
%
\documentclass[
]{article}
\usepackage{amsmath,amssymb}
\usepackage{iftex}
\ifPDFTeX
  \usepackage[T1]{fontenc}
  \usepackage[utf8]{inputenc}
  \usepackage{textcomp} % provide euro and other symbols
\else % if luatex or xetex
  \usepackage{unicode-math} % this also loads fontspec
  \defaultfontfeatures{Scale=MatchLowercase}
  \defaultfontfeatures[\rmfamily]{Ligatures=TeX,Scale=1}
\fi
\usepackage{lmodern}
\ifPDFTeX\else
  % xetex/luatex font selection
\fi
% Use upquote if available, for straight quotes in verbatim environments
\IfFileExists{upquote.sty}{\usepackage{upquote}}{}
\IfFileExists{microtype.sty}{% use microtype if available
  \usepackage[]{microtype}
  \UseMicrotypeSet[protrusion]{basicmath} % disable protrusion for tt fonts
}{}
\makeatletter
\@ifundefined{KOMAClassName}{% if non-KOMA class
  \IfFileExists{parskip.sty}{%
    \usepackage{parskip}
  }{% else
    \setlength{\parindent}{0pt}
    \setlength{\parskip}{6pt plus 2pt minus 1pt}}
}{% if KOMA class
  \KOMAoptions{parskip=half}}
\makeatother
\usepackage{xcolor}
\usepackage{color}
\usepackage{fancyvrb}
\newcommand{\VerbBar}{|}
\newcommand{\VERB}{\Verb[commandchars=\\\{\}]}
\DefineVerbatimEnvironment{Highlighting}{Verbatim}{commandchars=\\\{\}}
% Add ',fontsize=\small' for more characters per line
\newenvironment{Shaded}{}{}
\newcommand{\AlertTok}[1]{\textcolor[rgb]{1.00,0.00,0.00}{\textbf{#1}}}
\newcommand{\AnnotationTok}[1]{\textcolor[rgb]{0.38,0.63,0.69}{\textbf{\textit{#1}}}}
\newcommand{\AttributeTok}[1]{\textcolor[rgb]{0.49,0.56,0.16}{#1}}
\newcommand{\BaseNTok}[1]{\textcolor[rgb]{0.25,0.63,0.44}{#1}}
\newcommand{\BuiltInTok}[1]{\textcolor[rgb]{0.00,0.50,0.00}{#1}}
\newcommand{\CharTok}[1]{\textcolor[rgb]{0.25,0.44,0.63}{#1}}
\newcommand{\CommentTok}[1]{\textcolor[rgb]{0.38,0.63,0.69}{\textit{#1}}}
\newcommand{\CommentVarTok}[1]{\textcolor[rgb]{0.38,0.63,0.69}{\textbf{\textit{#1}}}}
\newcommand{\ConstantTok}[1]{\textcolor[rgb]{0.53,0.00,0.00}{#1}}
\newcommand{\ControlFlowTok}[1]{\textcolor[rgb]{0.00,0.44,0.13}{\textbf{#1}}}
\newcommand{\DataTypeTok}[1]{\textcolor[rgb]{0.56,0.13,0.00}{#1}}
\newcommand{\DecValTok}[1]{\textcolor[rgb]{0.25,0.63,0.44}{#1}}
\newcommand{\DocumentationTok}[1]{\textcolor[rgb]{0.73,0.13,0.13}{\textit{#1}}}
\newcommand{\ErrorTok}[1]{\textcolor[rgb]{1.00,0.00,0.00}{\textbf{#1}}}
\newcommand{\ExtensionTok}[1]{#1}
\newcommand{\FloatTok}[1]{\textcolor[rgb]{0.25,0.63,0.44}{#1}}
\newcommand{\FunctionTok}[1]{\textcolor[rgb]{0.02,0.16,0.49}{#1}}
\newcommand{\ImportTok}[1]{\textcolor[rgb]{0.00,0.50,0.00}{\textbf{#1}}}
\newcommand{\InformationTok}[1]{\textcolor[rgb]{0.38,0.63,0.69}{\textbf{\textit{#1}}}}
\newcommand{\KeywordTok}[1]{\textcolor[rgb]{0.00,0.44,0.13}{\textbf{#1}}}
\newcommand{\NormalTok}[1]{#1}
\newcommand{\OperatorTok}[1]{\textcolor[rgb]{0.40,0.40,0.40}{#1}}
\newcommand{\OtherTok}[1]{\textcolor[rgb]{0.00,0.44,0.13}{#1}}
\newcommand{\PreprocessorTok}[1]{\textcolor[rgb]{0.74,0.48,0.00}{#1}}
\newcommand{\RegionMarkerTok}[1]{#1}
\newcommand{\SpecialCharTok}[1]{\textcolor[rgb]{0.25,0.44,0.63}{#1}}
\newcommand{\SpecialStringTok}[1]{\textcolor[rgb]{0.73,0.40,0.53}{#1}}
\newcommand{\StringTok}[1]{\textcolor[rgb]{0.25,0.44,0.63}{#1}}
\newcommand{\VariableTok}[1]{\textcolor[rgb]{0.10,0.09,0.49}{#1}}
\newcommand{\VerbatimStringTok}[1]{\textcolor[rgb]{0.25,0.44,0.63}{#1}}
\newcommand{\WarningTok}[1]{\textcolor[rgb]{0.38,0.63,0.69}{\textbf{\textit{#1}}}}
\usepackage{graphicx}
\makeatletter
\def\maxwidth{\ifdim\Gin@nat@width>\linewidth\linewidth\else\Gin@nat@width\fi}
\def\maxheight{\ifdim\Gin@nat@height>\textheight\textheight\else\Gin@nat@height\fi}
\makeatother
% Scale images if necessary, so that they will not overflow the page
% margins by default, and it is still possible to overwrite the defaults
% using explicit options in \includegraphics[width, height, ...]{}
\setkeys{Gin}{width=\maxwidth,height=\maxheight,keepaspectratio}
% Set default figure placement to htbp
\makeatletter
\def\fps@figure{htbp}
\makeatother
\setlength{\emergencystretch}{3em} % prevent overfull lines
\providecommand{\tightlist}{%
  \setlength{\itemsep}{0pt}\setlength{\parskip}{0pt}}
\setcounter{secnumdepth}{-\maxdimen} % remove section numbering
\ifLuaTeX
  \usepackage{selnolig}  % disable illegal ligatures
\fi
\IfFileExists{bookmark.sty}{\usepackage{bookmark}}{\usepackage{hyperref}}
\IfFileExists{xurl.sty}{\usepackage{xurl}}{} % add URL line breaks if available
\urlstyle{same}
\hypersetup{
  hidelinks,
  pdfcreator={LaTeX via pandoc}}

\author{}
\date{}

\begin{document}

目录

收起

一、列出Excel VBA工程或Access数据库中引用的所有DLL库或ActiveX控件

二、DLL链接库或控件的前期引用与后期引用

三、通过VBA代码来动态添加前期引用

1)如果文件位置固定,可以使用AddFromFile方法

2)使用AddFromGuid方法动态添加引用

3)AddFromGuid方法使用方法如下:

4)AddFromGuid一些使用示例

5)注意事项:

6)Access使用动态引用开发的成品系统

\textbf{用代码写代码,注意:本文需要有一定的VBA基础及Excel或Access开发经验。}

**如阅读有些吃力,可以先收藏及关注我们

\href{//www.zhihu.com/people/61181a729ad96882fd24553e3f54ac6f}{@小辣椒高效Office}

**

\subsection{一、列出Excel
VBA工程或Access数据库中引用的所有DLL库或ActiveX控件}\label{ux4e00ux5217ux51faexcel-vbaux5de5ux7a0bux6216accessux6570ux636eux5e93ux4e2dux5f15ux7528ux7684ux6240ux6709dllux5e93ux6216activexux63a7ux4ef6}

可以使用下面的参考代码

作者:\href{https://link.zhihu.com/?target=http\%3A//www.office-cn.net/thread-38113-1-1.html}{Office交流网}
fans.net版主

\begin{Shaded}
\begin{Highlighting}[]
\NormalTok{Option Compare Database}
\NormalTok{Dim blnMark As Boolean}
\NormalTok{Dim intMark As Integer}
\NormalTok{\textquotesingle{}当指向 Application 对象的变量超出范围时,它所表示的 Microsoft Access 实例也将关闭。}
\NormalTok{\textquotesingle{}所以,必须在模块级说明这个变量。}
\NormalTok{Dim appAccess As Access.Application}

\NormalTok{\textquotesingle{}获取其他Access数据库中引用的所有类库和控件}
\NormalTok{Function GetRefrencesString(strDB As String) As String}
\NormalTok{On Error GoTo Err\_GetRefrencesString}

\NormalTok{Dim i As Integer}

\NormalTok{    Set appAccess = CreateObject("Access.Application")}
\NormalTok{        appAccess.OpenCurrentDatabase strDB}
    
\NormalTok{    For i = 1 To appAccess.Application.References.Count}
\NormalTok{        GetRefrencesString = GetRefrencesString \& appAccess.Application.References(i).Name \& \_}
\NormalTok{                        ":" \& vbTab \& appAccess.Application.References(i).FullPath \& vbCrLf}
\NormalTok{    Next}
    
\NormalTok{        appAccess.CloseCurrentDatabase}
\NormalTok{    Set appAccess = Nothing}
   
\NormalTok{Exit\_GetRefrencesString:}
\NormalTok{    Exit Function}

\NormalTok{Err\_GetRefrencesString:}
\NormalTok{    Set appAccess = Nothing}
\NormalTok{    MsgBox Err.Description}
\NormalTok{    Resume Exit\_GetRefrencesString}

\NormalTok{End Function}
\end{Highlighting}
\end{Shaded}

\begin{Shaded}
\begin{Highlighting}[]
\NormalTok{\textquotesingle{}列出程序中引用的所有类库和控件}
\NormalTok{Function ListRefrences() As String}
\NormalTok{Dim i As Integer}
\NormalTok{For i = 1 To Application.References.Count}
\NormalTok{     ListRefrences = ListRefrences \& Application.References(i).Name \& ":" \& vbTab \_}
\NormalTok{                            \& Application.References(i).FullPath \& vbCrLf}
\NormalTok{Next}
\NormalTok{End Function}
\end{Highlighting}
\end{Shaded}

\subsection{二、DLL链接库或控件的前期引用与后期引用}\label{ux4e8cdllux94feux63a5ux5e93ux6216ux63a7ux4ef6ux7684ux524dux671fux5f15ux7528ux4e0eux540eux671fux5f15ux7528}

vba要引用第三方库或控件,就要先添加DLL链接库或控件的引用,有两种引用方式

\textbf{1)前期引用}

前期引用,是在``工具''菜单下的``引用''命令中添加需要引用的库,如下图所示。

\includegraphics{https://pic3.zhimg.com/v2-53559ceae09673d321f36022454f894a/_b.jpg}

\textbf{2)后期引用}

则是使用类似这样的语句创建:

直接用 CreateObject(``Scripting.Dictionary'')语句 这就是后期引用。

如引用Excel对象

\begin{Shaded}
\begin{Highlighting}[]
\NormalTok{Sub GetExcel()}
\NormalTok{\textquotesingle{}Bind to an existing or created instance of Microsoft Excel}
\NormalTok{Dim objApp As Object}
\NormalTok{\textquotesingle{}Attempt to bind to an open instance}
\NormalTok{On Error Resume Next}
\NormalTok{Set objApp = GetObject(, "Excel.Application")}

\NormalTok{If Err.Number \textless{}\textgreater{} 0 Then}
\NormalTok{    \textquotesingle{}Could not get instance, so create a new one}
\NormalTok{    Err.Clear}
\NormalTok{    Set objApp = CreateObject("Excel.Application")}
\NormalTok{    With objApp}
\NormalTok{        .Visible = True}
\NormalTok{        .Workbooks.Add}
\NormalTok{    End With}
\NormalTok{End If}
\NormalTok{End Sub}
\end{Highlighting}
\end{Shaded}

\textbf{两者区别:}

1)前期引用可以直接列出成员属性、方法列表

2)后期引用不可以直接列出成员属性、方法列表。但不会因为控件或DLL不存在,程序一打开就出现引用丢失的情况,不是所有对象或控件均支持后期引用的(tmtony)。另有些有界面和窗体的控件不太方便使用后期引用。

\subsection{三、通过VBA代码来动态添加前期引用}\label{ux4e09ux901aux8fc7vbaux4ee3ux7801ux6765ux52a8ux6001ux6dfbux52a0ux524dux671fux5f15ux7528}

前期引用一般都是通过手动添加要引用库或ActiveX(OCX)控件的方式实现,但也可以使用vba代码自动添加需要的引用库,这里就要用到
References对象的AddFromGuid方法或者AddFromFile方法。

\subsubsection{\texorpdfstring{\textbf{1)如果文件位置固定,可以使用AddFromFile方法}}{1)如果文件位置固定,可以使用AddFromFile方法}}\label{1uxff09ux5982ux679cux6587ux4ef6ux4f4dux7f6eux56faux5b9aux53efux4ee5ux4f7fux7528addfromfileux65b9ux6cd5}

摘自:\href{https://link.zhihu.com/?target=http\%3A//www.office-cn.net/forum.php\%3Fmod\%3Dredirect\%26goto\%3Dfindpost\%26ptid\%3D83129\%26pid\%3D529320\%26fromuid\%3D75}{Office交流网}

\begin{Shaded}
\begin{Highlighting}[]
\NormalTok{可加载一个空窗体如下,注册控件,再启动窗体2打开实际功能。另外也将控件打包成安装包。}
\NormalTok{Private Sub Form\_Open(Cancel As Integer)}

\NormalTok{    DoCmd.RunCommand acCmdAppMinimize}
\NormalTok{    Me.Visible = False}

\NormalTok{    AutoRegFile "控件名"}

\NormalTok{    DoCmd.Close}
\NormalTok{    DoCmd.OpenForm "窗体2"}
\NormalTok{End Sub}

\NormalTok{\textquotesingle{}这是网上高手写的}
\NormalTok{Function AutoRegFile(FileName As String)}
\NormalTok{    Dim reged As Boolean}
\NormalTok{    Dim RegFile1 As String}
\NormalTok{    Dim RegFile2 As String}
\NormalTok{    Dim BeReg As String, strDtn As String, strDtn1 As String}
\NormalTok{    Dim ref As Reference}

\NormalTok{    Dim RetVal}
\NormalTok{    BeReg = CurrentProject.Path \& "\textbackslash{}ocx\textbackslash{}" \& FileName          \textquotesingle{}控件存放位置,例子中是放在工程当前目录下ocx子目录}
\NormalTok{    strDtn = Environ("windir") \& "\textbackslash{}system\textbackslash{}" \& FileName           \textquotesingle{}返回系统路径}
\NormalTok{    strDtn1 = Environ("windir") \& "\textbackslash{}system32\textbackslash{}" \& FileName           \textquotesingle{}返回系统路径}
\NormalTok{    On Error Resume Next}

\NormalTok{    RegFile1 = Environ("windir") \& "\textbackslash{}system\textbackslash{}regsvr32.exe "}
\NormalTok{    RegFile2 = Environ("windir") \& "\textbackslash{}system32\textbackslash{}regsvr32.exe "}

\NormalTok{    If Dir(RegFile1) \textless{}\textgreater{} "" Or Dir(RegFile2) \textless{}\textgreater{} "" Then}
\NormalTok{        If Dir(RegFile1) \textless{}\textgreater{} "" Then}
\NormalTok{            FileCopy BeReg, strDtn}
\NormalTok{            RegFile1 = RegFile1 \& "/s" \& " " \& strDtn}
\NormalTok{            RetVal = Shell(RegFile1, 1)}
\NormalTok{\textquotesingle{}            Set ref = References.AddFromFile(Environ("windir") \& "\textbackslash{}system\textbackslash{}" \& FileName)}
\NormalTok{        Else}
\NormalTok{            FileCopy BeReg, strDtn1}
\NormalTok{            RegFile2 = RegFile2 \& "/s" \& " " \& strDtn1}
\NormalTok{            RetVal = Shell(RegFile2, 1)}
\NormalTok{\textquotesingle{}            Set ref = References.AddFromFile(Environ("windir") \& "\textbackslash{}system32\textbackslash{}" \& FileName)    \textquotesingle{}设置引用}
\NormalTok{        End If}
\NormalTok{    Else}
\NormalTok{        MsgBox "找不到regsvr32.exe文件,你可能无法使用本软件!", vbCritical, "无法自动注册控件"}
\NormalTok{    End If}
\NormalTok{End Function}
\end{Highlighting}
\end{Shaded}

AddFromFile方法是通过添加具体的文件路径的方法来实现引用,这种方法缺点就是,不同的操作系统、不同用户的安装路径可能会不同,这样路径不一致,所以你做好的程序复制到用户或同事的电脑,可能会出错。

所以就要用到我们第2种方法:AddFromGuid方法

\subsubsection{\texorpdfstring{\textbf{2)使用AddFromGuid方法动态添加引用}}{2)使用AddFromGuid方法动态添加引用}}\label{2uxff09ux4f7fux7528addfromguidux65b9ux6cd5ux52a8ux6001ux6dfbux52a0ux5f15ux7528}

\textbf{AddFromGuid方法是直接根据全局唯一标识符字符串globally unique
identifier (GUID)
来添加引用,这种方法可以跨操作系统、跨版本都有效。一般知名厂家的DLL或OCX
的GUID是统一的。}

全局唯一标识符字符串(GUID)
是一个唯一标识符,它不会因为引用的版本号的变化而变化。所以通过AddFromGuid方法可以保证引用的准确性。不会像AddFromFile方法因为路径不同而出错!

\subsubsection{\texorpdfstring{\textbf{3)AddFromGuid方法使用方法如下:}}{3)AddFromGuid方法使用方法如下:}}\label{3uxff09addfromguidux65b9ux6cd5ux4f7fux7528ux65b9ux6cd5ux5982ux4e0b}

\begin{Shaded}
\begin{Highlighting}[]
\NormalTok{AddFromGuid(Guid, Major, Minor) As Reference}
\end{Highlighting}
\end{Shaded}

参数guid, major, minor分别表示引用的全局唯一标识符,major
主版本号和minor 小版本号。

AddFromGUID方法基于标识类型库的 GUID创建一个引用对象。参考对象。

\textbf{参数}

姓名 必需/可选 数据类型 描述

指导 必需的 细绳 标识类型库的 GUID。

主要的 必需的 长 参考的主要版本号。

次要的 必需的 长 参考的次要版本号。

GUID属性返回指定引用对象的GUID。如果您存储了GUID属性的值,则可以使用它来重新创建已损坏的引用。

如果您为主要和次要版本参数添加使用 0 的 GUID
引用,它将解析为对象库的最新安装版本。

\subsubsection{\texorpdfstring{\textbf{4)}AddFromGuid一些使用\textbf{示例}}{4)AddFromGuid一些使用示例}}\label{4uxff09addfromguidux4e00ux4e9bux4f7fux7528ux793aux4f8b}

以下示例根据用户系统上的 GUID重新创建对Microsoft Scripting Runtime 1.0
版的引用。

References.AddFromGuid "\{420B2830-E718-11CF-893D-00A0C9054228\}", 1, 0

以下示例添加了对Microsoft Excel
对象库的引用,但不知道当前安装的是哪个版本。

References.AddFromGuid "\{00020813-0000-0000-C000-000000000046\}", 0, 0

为了方便查看这些参数,我们可以通过先手动添加引用的库(如Excel常用的字典库
Scripting.Dictionary),然后用代码获取它的相关属性:

\begin{Shaded}
\begin{Highlighting}[]
\NormalTok{Sub GetRefInfo()}
\NormalTok{    Dim ref As Reference}
\NormalTok{        \textquotesingle{}遍历显示所有引用的相关信息}
\NormalTok{        For Each ref In ThisWorkbook.VBProject.References}
\NormalTok{            With ref}
\NormalTok{                Debug.Print "引用的名称:" \& .Name}
\NormalTok{                Debug.Print "引用的路径:" \& .FullPath}
\NormalTok{                Debug.Print "GUID:" \& .GUID}
\NormalTok{                Debug.Print "Major:" \& .Major}
\NormalTok{                Debug.Print "Minor:" \& .Minor}
\NormalTok{                Debug.Print "描述" \& .Description}
\NormalTok{                i = i + 1}
\NormalTok{            End With}
\NormalTok{        Next}
\NormalTok{End Sub}
\end{Highlighting}
\end{Shaded}

\textbf{如 Scripting.Dictionary 对应引用是Microsoft Scripting Runtime}\\
\textbf{GUID: \{420B2830-E718-11CF-893D-00A0C9054228\}}\\
\textbf{major: 1}\\
\textbf{minor: 0}

这样就可以通过VBA代码自动 添加Microsoft Scripting Runtime:

如果重复添加
会弹出``名称冲突的提示'',可以先用代码先去除已判断的引用,或判断引用是否存在,或直接跳过错误处理:

\begin{Shaded}
\begin{Highlighting}[]
\NormalTok{Sub AddRef()}
\NormalTok{    On Error Resume Next \textquotesingle{}或先 用 application.References.Remove  去掉引用}
\NormalTok{    Dim ref As Reference}
\NormalTok{    Set ref = ThisWorkbook.VBProject.References.AddFromGuid("\{420B2830{-}E718{-}11CF{-}893D{-}00A0C9054228\}", 1, 0)}
\NormalTok{End Sub}
\end{Highlighting}
\end{Shaded}

\textbf{以下是一些常用的对象的GUID及VBA代码动态添加前期引用的代码:}

\begin{Shaded}
\begin{Highlighting}[]
\NormalTok{    \textquotesingle{}excel对象}
\NormalTok{    Set ref = ThisWorkbook.VBProject.References.AddFromGuid("\{00020813{-}0000{-}0000{-}C000{-}000000000046\}", 1, 9)}
\NormalTok{    \textquotesingle{}word对象}
\NormalTok{    Set ref = ThisWorkbook.VBProject.References.AddFromGuid("\{00020905{-}0000{-}0000{-}C000{-}000000000046\}", 8, 7)}
\NormalTok{    \textquotesingle{}ppt对象}
\NormalTok{    Set ref = ThisWorkbook.VBProject.References.AddFromGuid("\{91493440{-}5A91{-}11CF{-}8700{-}00AA0060263B\}", 2, 12)}
\NormalTok{    \textquotesingle{}WinHttp对象}
\NormalTok{    Set ref = ThisWorkbook.VBProject.References.AddFromGuid("\{662901FC{-}6951{-}4854{-}9EB2{-}D9A2570F2B2E\}", 5, 1)}
\NormalTok{    \textquotesingle{}MSHTML对象}
\NormalTok{    Set ref = ThisWorkbook.VBProject.References.AddFromGuid("\{3050F1C5{-}98B5{-}11CF{-}BB82{-}00AA00BDCE0B\}", 4, 0)}
\NormalTok{    \textquotesingle{}VBIDE对象}
\NormalTok{    Set ref = ThisWorkbook.VBProject.References.AddFromGuid("\{0002E157{-}0000{-}0000{-}C000{-}000000000046\}", 5, 3)}
\end{Highlighting}
\end{Shaded}

\subsubsection{5)注意事项:}\label{5uxff09ux6ce8ux610fux4e8bux9879}

1、在引用未成功添加之前,不要引用库中的相关函数及方法\\
2、Access 或 Excel
用户窗体可以用一个干净的启动窗体先实现代码动态引用,待引用成功后再打开使用引用库的其它窗体。

\subsubsection{6)Access使用动态引用开发的成品系统}\label{6uxff09accessux4f7fux7528ux52a8ux6001ux5f15ux7528ux5f00ux53d1ux7684ux6210ux54c1ux7cfbux7edf}

我们的Access通用开发平台核心库是DevLib.mde
(改名为\href{https://link.zhihu.com/?target=http\%3A//Devlib.net}{http://Devlib.net}
)
然后在Main.mdb中使用启动代码动态添加对\href{https://link.zhihu.com/?target=http\%3A//Devlib.net}{http://Devlib.net}的引用,这样无论程序复制到哪个路径,均能正常运行。

\href{https://link.zhihu.com/?target=http\%3A//www.office-cn.net/product/2.html}{}

\href{https://link.zhihu.com/?target=http\%3A//www.office-cn.net/product/2.html}{Access通用开发平台标准版(免费使用)
- Access开发平台 - Office交流网}

\includegraphics{https://pic2.zhimg.com/v2-2bb71ff974318f5d8bf0222bbfc9d091/_b.jpg}

\includegraphics{https://pic2.zhimg.com/v2-353af8777d51d7364ebc47ca40b716d9/_b.jpg}

\includegraphics{https://pic4.zhimg.com/v2-19a87e370e7c0c31f89c9bf8d77765fb/_b.jpg}

如果您对Excel VBA
及Access开发部门级或企业级管理系统有兴趣,也可咨询我或私聊:

\includegraphics{https://pica.zhimg.com/v2-84c311f9235b1da27cc3af1ce7b643c2_l.jpg?source=f2fdee93}

小辣椒高效Office

19 次咨询

5.0

18326 次赞同

去咨询

**希望这篇文章对您有用,可点赞 收藏 关注 三连

\href{//www.zhihu.com/people/61181a729ad96882fd24553e3f54ac6f}{@小辣椒高效Office}

**

本文转自
\url{https://zhuanlan.zhihu.com/p/524202003?utm_id=0},如有侵权,请联系删除。

\end{document}
